\documentclass[10pt,a4paper]{scrartcl}
\usepackage[utf8]{inputenc}
\usepackage[english,ngerman]{babel}
\usepackage{amsmath}
\usepackage{amsfonts}
\usepackage{amssymb}
\usepackage{graphicx}
\usepackage{geometry}
\usepackage{setspace}
\usepackage{tabularx}

\sloppy

\geometry{a4paper, portrait, left=2cm, right=2cm, top=2cm, bottom=2cm}

\renewcommand*{\titlepagestyle}{empty}

\title{Einverständniserklärung}
\subtitle{Studie im Rahmen der HCI-Veranstaltung im Sommersemester 2016}


\date{\vspace{-5ex}}

\begin{document}
\maketitle
\pagestyle{empty}

\vspace{0.5 cm}

\noindent
\textbf{Um Ihr Einverständnis zur Teilnahme an dieser Studie zu erklären, bitten wir Sie dieses Dokument zu lesen und zu unterzeichnen.}

\vspace{2 ex}
\noindent
Sie sind dazu eingeladen an einer explorativen Studie zur Mensch-Maschine-Interaktion teilzunehmen. Die Studie wird von Studenten der Christian-Albrechts-Universität zu Kiel im Rahmen der Veranstaltung „Quantitative Methods in Human-Computer Studies“ unter der Leitung von Prof. Dr. habil. Ansgar Scherp durchgeführt.

\subsection*{Ziel der Studie}
Diese Studie wird durchgeführt um Erkenntnisse zur Effizienz, Effektivität und Zufriedenheit verschiedener Eingabemethoden bei der Navigation in Applikationen auf mobilen Endgeräten (Smartphones und Tablets) zu gewinnen. Konkret müssen Sie sich spielerisch durch ein Labyrinth bewegen. Die Aufgaben spiegeln für bestimmte Arten von Applikationen typische Interaktionen wider. Zu Beginn der Studie wird Ihnen alles erklären.
Ihre Eingaben werden über interne Mechanismen der Applikation aufgezeichnet, um eine genauere Auswertung zu ermöglichen.

\subsection*{Anonymität und Vertraulichkeit}
Bitte beachten Sie, dass natürlich alle Aufzeichnungen und Antworten, die während der Sitzung anfallen, streng vertraulich behandelt werden. Darüber hinaus wird Ihre Identität anonym verbleiben und in keiner Weise mit den Forschungsdaten verknüpft. Die Aufzeichnungen werden nicht an Dritte weitergegeben und nur im Rahmen dieser Studie verwendet.

\subsection*{Zustimmung zur Teilnahme}
Mit der Unterzeichnung dieses Formulars stimme ich Folgendem zu:
\begin{itemize}
	\item Ich habe verstanden, dass die Teilnahme an dieser Studie freiwillig ist und ich jederzeit die weitere Teilnahme abbrechen kann.
	\item Ich erlaube die Aufzeichnung meiner Eingaben über die Applikation zum Zwecke der Analyse im Rahmen der oben formulierten Ziele dieser Studie.
	\item Ich habe dieses Dokument gelesen und verstanden. Alle meine Fragen bezüglich der Studie wurden beantwortet und ich bin damit einverstanden an dieser Studie teilzunehmen.
\end{itemize}

\subsection*{Zustimmung widerrufen}
Sollten Sie Ihre Zustimmung widerrufen wollen, so können Sie dies mit einer E-Mail an \textbf{hci.study.2016@gmail.com} und Nennung Ihrer ID machen. Alle von Ihnen erhobenen Daten werden dann gelöscht. Uns ist es ohne diese Identifikationsnummer nicht möglich, die Daten mit Ihnen in Verbindung zu bringen.

\vspace{3 cm} 
\begin{tabular}{p{7cm}p{.5cm}l}
	\dotfill \\ 
	Ort, Datum
\end{tabular}
\hfill 
\begin{tabular}{p{7cm}p{.5cm}l}
	\dotfill \\ 
	Unterschrift
\end{tabular}

\newpage

\maketitle

\subsection*{Zustimmung widerrufen}
Sollten Sie Ihre Zustimmung widerrufen wollen, so können Sie dies mit einer E-Mail an \textbf{hci.study.2016@gmail.com} und Nennung Ihrer ID machen. Alle von Ihnen erhobenen Daten werden dann gelöscht. Uns ist es ohne diese Identifikationsnummer nicht möglich, die Daten mit Ihnen in Verbindung zu bringen.

\vspace{2 cm}
\noindent
Ihre persönliche ID: \dotfill\\

\end{document}
