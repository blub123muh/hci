\documentclass{beamer}
\usepackage[utf8]{inputenc}
\AtBeginSection[]
{
        \begin{frame}
                \frametitle{Table of Contents}
                \tableofcontents[currentsection]
        \end{frame}
}
\begin{document}
% 20-30

\title{Swiping Burger}
\author{Florian, Steffen, Felix, Lukas}

\frame{\maketitle}

\section{Expose}
\begin{frame}
        \frametitle{Expose}

\end{frame}

\begin{frame}
        \frametitle{Hypotheses}
        \begin{itemize}
                \item Usage of Hamburger Menu navigation requires constant time to travel a path.
                \item Usage of Swipe navigation
                        results in growing time depending on the distance.
                \item There is some point of intersection,
                        where the required navigation time of Swipe exceeds the required navigation time of Hamburger Menu.
        \end{itemize}

\end{frame}


\section{Design}
\subsection{Within-, Between-group or Split-plot}
\begin{frame}
        \frametitle{Within-, Between-group or Split-plot}
        \begin{itemize}
                \item Between group treatment among navigation methods
                \item Within group design over the length of the navigation paths
        \end{itemize}
        $\rightarrow$ Split-plot design.
\end{frame}

\subsection{Independent and Dependent Variables}
\begin{frame}
        \frametitle{Independent / Dependent Variables}
        \begin{block}{Independent Variables}
                \begin{itemize}
                        \item Navigation Method
                        \item Navigation Distance
                \end{itemize}
        \end{block}
        \begin{block}{Dependent Variables}
                \begin{itemize}
                        \item Time 
                        \item User satisfaction (Questionnaires)
                \end{itemize}
        \end{block}
\end{frame}

\subsection{Participants and Location}
\begin{frame}
        \frametitle{Participant Acquisition and where to investigate}
        \begin{block}{Field study}
                \begin{description}
                        \item[Why?] Participants should not get bored.
                                Experiment is not too long.
                        \item[Where?] Mensa: because student's are used to smartphones.
                        \item[When?] Not during meal times to avoid distraction.
                \end{description}
        \end{block}
\end{frame}

\subsection{Questionnaires}
\begin{frame}
        \frametitle{Questionnaires}
        \begin{block}{After each distance block}
                \begin{itemize}
                        \item `` Ich konnte die Aufgaben gut lösen. ''
                        \item `` Die Erreichung des Ziels war mir zu umständlich.''
                \end{itemize}
        \end{block}
        \begin{block}{At the very end}
                \begin{itemize}
                        \item ``Die Umgebung hat mich beim Bearbeiten der Aufgaben gestört''
                        \item ``Die Dauer des Experiments war mir zu hoch?''
                        \item ``Die Bedienung des mobilen Geräts war intuitiv''
                        \item ``Die Art der Navigation hat mir zugesagt''
                \end{itemize}
        \end{block}
\end{frame}

\section{Analysis Methods}
\begin{frame}
        \frametitle{Hamburger Menu Navigation over distances $d$}
        Consider the normally distributed random variables $X_d$ with $\bar x$ and
        $\sigma_x^2$ being the time required for traveling $d$ pages using Hamburger
        Menu navigation.
        We hypothize that
        \begin{align*}
                H_{0, \text{burger}}: \bar x_0 = \bar x_1 = \cdots = \bar x_d
        \end{align*}
        Since we can assume equal variances,
        we evaluate the hypothesis using Repeated Measures {ANOVA}.
        In case we do not find significant differences, we can use the overall mean
        $\bar x$ for comparision with the Swipe navigation. 
\end{frame}

\begin{frame}
        \frametitle{Swipe Menu Navigation over distances $d$}
        Consider normally
        distributed random variables $Y_d$ with means $\bar y_d$ and variances $\sigma_{y,d}^2$ as the
        time required for traveling $d$ pages using Swipe navigation. 
        \begin{align*}
                H_{0, \text{swipe}}: \bar y_0 = \bar y_1= \cdots = \bar y_d
        \end{align*}
        Since we can not assume homogeneity of variance in this case,
        we evaluate the hypothesis using the Friedmann test. 
\end{frame}

\begin{frame}
        \frametitle{Swipe Navigation vs Hamburger Navigation}
        In case we can reject $H_{0, \text{swipe}}$ and can not reject $H_{0, \text{burger}}$,
        we continue with comparing per-distance means $\bar y_d$ of Swipe navigation
        with overall mean $\bar x$ of Hamburger Menu navigation.
        \begin{align*}
                H_{0,d} : \bar y_d = \bar x \;\forall d
        \end{align*}
        We evaluate these hypotheses using the Welch's $t$-test,
        because we can not assume equal variances.
        In the end we (hopefully) find the distance $d$,
        for which the the time required with Swipe navigation exceeds the time required with Hamburger Menu navigation.
\end{frame}
\end{document}
