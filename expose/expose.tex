%        File: expose.tex
%     Created: Sun May 01 10:00 PM 2016 C
% Last Change: Sun May 01 10:00 PM 2016 C
%
\documentclass[a4paper]{article}
\author{Steffen Goos, Florian Mai, Lukas Galke, Felix Paur}
\title{Analysis of the efficiency and user satisfaction of an app menu}

\begin{document}
\maketitle
We analyze how the position of a menu on a website affects the efficiency and the user satisfaction.\\
Most websites have at least one menu and that menu has to be positioned.
So it is of interest whether its position affects the user's satisfaction and efficiency.
There are also many confusing websites.
In order to make information easily accessible, it is a very basic but interesting question
whether the position of a menu affects the user experience.\\
In this work, we will only consider websites with one menu,  which is either in english or german.
Since those languages are read from top left to bottom right,
we expect that for someone used to read in that fashion a top left position will have the most positive influence
on the users efficiency and satisfaction.\\
To verify or falsify this hypothesis,  we will design websites with the same menu structure but different menu positions.
Then we will have people fulfill tasks on those websites and evaluate their results.
\end{document}

