%        File: expose.tex
%     Created: Sun May 01 10:00 PM 2016 C
% Last Change: Sun May 01 10:00 PM 2016 C
%
\documentclass[a4paper]{article}
\author{Steffen Goos, Florian Mai, Lukas Galke, Felix Paur}
%\title{Analysis of the efficiency and user satisfaction of an app menu}
\title{Analysis of the efficiency and user satisfaction of a menu and swipe gestures in the context of a survey app}

\begin{document}
\maketitle
For several years, the swipe gesture has been a very-well established way to navigate among adjacent pages on websites as well as in native
apps on a mobile device. The users appreciate its ease and its comfort of use. However, it can only be applied in scenarios where
the content is provided in a sequential way, e.g. in a picture gallery, in an e-book reader, or when filling in a form. In this paper, we
examine the swipe gesture's suitability for navigating through a set of question in a survey. Although typically the user progresses 
through the questions sequentially, it is a common case that they want to revise their answers later due to a change of mind or a missunderstanding
of the question, for example. We conjecture that navigating back using swipe gestures is unsatisfactory and time-costly as the distance in pages
increases. In order to research this question, we conduct a study in which we compare the use of swipe gestures to the use of the also
well-established hamburger menu. We design tasks that capture the use cases described above. We alter the distance in pages
that has to be covered in order to complete the task. For each task, we measure the efficiency. Furthermore, we assess the user satisfaction through a questionnaire. We expect
both measures to be independent of the distance when the menu is used. When the swipe gesture is used, however, we expect both measures to decrease
as the distance increases. We suspect that the swipe gestures do better than the menu when the distance is small, but we also think that this
this relation flips at a number of pages that is realistic for surveys. Hence, the results of our study are of practical importance.
%We analyze how the position of a menu on a website affects the efficiency and the user satisfaction.\\
%Most websites have at least one menu and that menu has to be positioned.
%So it is of interest whether its position affects the user's satisfaction and efficiency.
%There are also many confusing websites.
%In order to make information easily accessible, it is a very basic but interesting question
%whether the position of a menu affects the user experience.\\
%In this work, we will only consider websites with one menu,  which is either in english or german.
%Since those languages are read from top left to bottom right,
%we expect that for someone used to read in that fashion a top left position will have the most positive influence
%on the users efficiency and satisfaction.\\
%To verify or falsify this hypothesis,  we will design websites with the same menu structure but different menu positions.
%Then we will have people fulfill tasks on those websites and evaluate their results.
\end{document}

